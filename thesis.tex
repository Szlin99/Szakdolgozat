\documentclass[a4paper,12pt,oneside]{report}
\usepackage[magyar]{babel}
\usepackage[utf8]{inputenc}
\usepackage{t1enc}
\def\magyarOptions{defaults=hu-min}
\usepackage{cite}
\usepackage{amsmath,amssymb,amsfonts}
\usepackage{algorithmic}
\usepackage{graphicx}
\usepackage{textcomp}
\usepackage[dvipsnames]{xcolor}
%\usepackage{booktabs}
\usepackage{tabularray}
\usepackage{listings}
\lstset
{
	framexleftmargin=10mm,
	frame=shadowbox,
	rulesepcolor=\color{olive8},
	language=C,
	numbers=left,
	stepnumber=1,
	showstringspaces=false,
	tabsize=4,
	breaklines=true,
	breakatwhitespace=false,
	basicstyle=\footnotesize\ttfamily
}
\frenchspacing
\linespread{1.25}
\UseTblrLibrary{booktabs}
\usepackage[left=2.5cm,top=4cm,right=2.5cm,bottom=4.5cm,headsep=2cm]{geometry}

\usepackage{titlesec, blindtext, color}
\definecolor{gray75}{gray}{0.75}

\usepackage{fancyhdr}

\usepackage{etoolbox}
\patchcmd{\chapter}{\thispagestyle{plain}}{\thispagestyle{fancy}}{}{}

\begin{document}

\pagestyle{fancy}
\fancyhead[]{}
\fancyhead[C]{\thepage}
\fancyfoot[]{}
\fancyfoot[]{}
\renewcommand{\headrulewidth}{0pt}
\renewcommand{\footrulewidth}{0pt}

\titleformat{\chapter}[hang]{\normalsize\bfseries}{\color{Black}\thechapter .\hspace{10pt}}{0pt}{\normalsize\bfseries\color{Black}\MakeUppercase}

\titleformat{\section}[hang]{\normalsize\bfseries}{\hspace{0pt} \thesection .\hspace{10pt}}{0pt}{\normalsize\bfseries\MakeUppercase}

\titleformat{\subsection}[hang]{\normalsize\bfseries}{\hspace{0pt} \thesubsection .\hspace{10pt}}{0pt}{\normalsize\bfseries}

\title{Szakdolgozat}

\author{Tórizs Dániel György}

\date{}

\maketitle

\tableofcontents

\chapter{Bevezetés}

Az ezt megelőző félévekben elkezdett okosotthon projekt folytatása az idei évi is, amely jelenleg egy Raspberry PI-t tartalmaz.
Idén ennek a megismerésével foglalkoztam, valamint a node-REDdel. Ezek bemutatását tartalmazza majd ez a dokumentáció.

Az alaphoz egy Raspberry PI-t használtam, ezt az eszközt az Egyesült Királyságban fejlesztették ki oktatási célokra, igazából
egy bankkártya méretű számítógép amely különböző Linux operációs rendszerekkel képes üzemelni. A lapkán megtalálható egy 700 MHz-es 
processzor és egy GPU egyes verziókban 256, míg másikakban 512 MB RAM található. A processzor egy régebbi okostelefonéval egyenlő.
A GP több felbontást támogat eszek közül néhány: 640x350,1024x768, 1280x720, 1920x1080.
Különböző operációs rendszerek elérhetőek hozzá:
\begin{itemize}

\item{Raspbian:Debian Raspberryre optimalizálva.}

\item{Minibian:Az előzőhöz hasonló, butított os.}

\item{Pidora:Fedora optimalizált os.}

\item{Risc OS:Nem Linux alapú operációs rendszer.}

\item{Retropie:Régi konzolok emulálásához fejlesztett rendszer.}

\item{Windows 10 IOT Core}

\end{itemize}

Különböző felhasználási területei vannak:
\begin{itemize}

\item{Játékszerverek.}
	
\item{Nyomtató szerver.}
	
\item{Háttérvilágítás vezérlés.}
	
\item{Robot vezérlés.}
	
\item{Interaktív játékok.}
	
\item{Kommunikációs szerver.}
	
\end{itemize}

A kis fogyasztási igénye miatt tökéletes például szerverek készíatéséhez.

Néhány Raspberry modell:
\begin{itemize}

	\item{PI Model B}

	Ez a modell egy 700 MHz-es ARM1176JZF-S processzorral rendelkezik, ami mellett egy Broadcom VideoCore IV-es videókártya is
	megtalálható. A lapon még 512 MB RAM-ot is elhelyeztek. Portok tekintetében két USB 2.0, egy HDMI valamint egy 3.5mm-es Jack
	is helyet kapott. Valamint van rajta 8 GPIO pin, egy UART, egy I2C bus, SPI, +3.3V, +5V, és föld.
	\begin{figure}[htbp]
		\centering
		\includegraphics[width=0.5\textwidth]{fig/modelb.png}
		\caption{Raspberry PI Model B}
		\label{fig-modelb}
	\end{figure}
		
	\item{PI Model A+}

	Ez a modell abban tér el az előzőtől, hogy csak 256 MB RAM-ot tartalmaz, valamint több belső csatlakozója van (40 GPIO, CSI,DSI).
	Külsö csatlakozókból pedig valamivel kevesebb 1 darab USB 2.0, 3.5mm Jack, HDMI.
	\begin{figure}[htbp]
		\centering
		\includegraphics[width=0.5\textwidth]{fig/aplusz.png}
		\caption{Raspberry PI Model A+}
		\label{fig-aplusz}
	\end{figure}
		
	\item{PI 3 Model B}

	Ez a modell 10-szer gyorsabb mint az eredeti Model B ezt a 64 bites 4 magos ARMv7 processzornak köszönheti ami 1.2 GHz-en üzemel,
	beépített Wlannal valamint Bluetooth-szal is felszerelték. A GPU egy kétmagos VideoCore IV, emellett 1 GB memória is helyet kapott.
	Belső csatlakozókat tekintve 40 darab pin van a lapon ebből 27 GPIO, valamint +3.3V és 5V és a föld.
	Külsö kimeneteknél HDMI, Composite RCA, 3.5mm Jack, 4 darab USB 2.0. CSI (Camera Serial Interface), DSI (Display Serial Interface).
	\begin{figure}[htbp]
		\centering
		\includegraphics[width=0.5\textwidth]{fig/3b.png}
		\caption{Raspberry PI 3 Model B}
		\label{fig-3b}
	\end{figure}
		
	\item{PI 4 Model B}

	A Raspberry PI 4 Model B egy BCM2711-es 4 magos, 64 bites processzorral van szerelve ami 1.5 GHz-en működik. Több memória változattal
	is rendelhető van 1, 2, illetve 4 GB-os változata is. Ebben is megtalálható Wlan, valamint Bluetooth 5.0. Ebben is 40 belső 
	csatlakazó található viszont a külsők fejlödtek USB tekintetében, 2 darab USB 3.0, valamint 2 darab USB 2.0 csatlakozó található
	rajta. 
	Megtalálható rajta 2 micro HDMI port ami 4Kp60 FPS-re képes, CSI,DSI.
	A 4 GB-os változatot irodai számítógépnek is ajánlják, mivel képes teljes értékű számítógépként működni.
	\begin{figure}[htbp]
		\centering
		\includegraphics[width=0.5\textwidth]{fig/4b.png}
		\caption{Raspberry PI 4 Model B}
		\label{fig-4b}
	\end{figure}
		
	\item{PI 400}

	A PI 400 egy billentyűzetbe épített számítógép, amely 4 GB RAM-mal, 8 magos 64 bites 5 GHz-es processzorral, valamint 3 USB portal 
	van felszerelve, ezeken kívül GigabitEthernet csatlakozóval, dupla micro HDMI kimenettel, illetve Bluetooth 5.0-val és 
	egy kétsávos Wi-Fi modullal érkezik. Ezek mellett azonban megmarad a 40 pines GPIO csatlakozó is.
	Ez a modell igazából egy könnyen hordozható számítógép.
	\begin{figure}[htbp]
		\centering
		\includegraphics[width=0.5\textwidth]{fig/pi400.png}
		\caption{Raspberry PI 400}
		\label{fig-pi400}
	\end{figure}

	\item{PI Zero 2 W}

	Ezt a Raspberry-t használtam a projektemhez.
	Ez a modell a PI 3-ban megtalálható BCM 2710A1-es processzort használja, amely 4 magos és 1 GHz-es, valamint 512 MB RAM-mal 
	van szerelve.
	Emelett Wlan és Bluetooth is található benne. MicroSD Kártyával bővíthető a tárhely, mint a legtöbb Raspberryben.
	Belső csatlakozók tekintetében 40 pin található.
	Külsö csatlakozók terén megtalálható egy micro HDMI, USB 2.0 OTG és egy USB táp és CSI.
	Ez a Raspberry nagyon jó választás robotokba és kamárákhoz.
	\begin{figure}[htbp]
		\centering
		\includegraphics[width=0.5\textwidth]{fig/pi0.png}
		\caption{Raspberry PI Zero 2 W}
		\label{fig-pi0}
	\end{figure}

	\begin{figure}[htbp]
		\centering
		\includegraphics[width=0.5\textwidth]{fig/gpio.png}
		\caption{Raspberry PI Zero 2 W Pin kiosztás}
		\label{fig-gpio}
	\end{figure}
		
	\end{itemize}

%\begin{itemize}

%	\item{Első bullet point}

%	\item{Második bullet point}

%	\item{Végül a harmadik}

%\end{itemize}

%És kódot is. Ilyenkor a LaTeX szépen sorszámozza nekünk, és egy keretet is tesz.
%(A sorszámozás, a keret, a behúzás stb. beállítható.)

%Ez egy kis térköz
%\medskip

%Így tudunk kódot beszúrni
%\begin{lstlisting}
%#include <stdio.h>
%int main()
%{
%	// printf() displays the string inside quotation
%	printf("Hello, World!");
%	return 0;
%}
%\end{lstlisting}

\chapter{Telepítés}

A telepítéshez a Raspberry PI Imagert használtam, ami egy elég egyszerű program.

\begin{figure}[htbp]
	\centering
	\includegraphics[width=0.5\textwidth]{fig/Imager.png}
	\caption{A telepítő program.}
	\label{fig-Imager}
\end{figure}

A programon bellül ki kell választanunk az operációs rendszert amit a Raspberry-n futtatni szeretnénk, ezen kívül a memóriakártyát, 
ami a háttértárunk lesz. Amikor ezzel megvagyunk még a beállításokban meg kell adnunk az eszköznek egy felhasználónevet és egy jelszót
valamint a router hozzáférési adatait.

\begin{figure}[htbp]
	\centering
	\includegraphics[width=0.5\textwidth]{fig/os.png}
	\caption{Operációs rendszerek.}
	\label{fig-os}
\end{figure}

Ha ezzel megvagyunk akkor elindul egy lassabb telepítési folyamat, amely a memóriakártyára helyezi az operációs rendszert, ezután pedig
egy ellenörzési folyamat, hogy minden rendben van-e.
Miután ezzel megvagyunk a memóriakártyát behelyezzük a Raspberry-be, ahol egy lassabb boot után elindul az eszköz (amikor ez megtörtént
akkor a led egy biztos ritmusra villog), ezután megkezdhetjük a hozzáférést az eszközhöz. 
A hozzáféréshez szükségünk van a router kezelőfelületére, ahol meg kell keresnünk az eszközünket, a legjobb ha adunk neki
egy saját nevet. Miután ez megvan a PuTTY segítségével hozzáférünk a Raspberry operációs rendszeréhez.

\begin{figure}[htbp]
	\centering
	\includegraphics[width=0.5\textwidth]{fig/putty.png}
	\caption{PuTTY beállítás.}
	\label{fig-putty}
\end{figure}

\chapter{node-RED}

A node-RED egy áramlás alapű fejlesztő környezet ahol grafikusan tudunk programot készíteni egy böngésző segítségével. Eredetileg az IBM
készítette. Az elkészítet programokat egy JSON fájlban tárolja. A programozás grafikusan történik úgynevezett "node"-ok használatával,
amelyek mindegyikének külön funkciója van, vannak amelyekhez adatot is kell rendelnünk. A node-okat kábelekkel kötjük össze amelyeken az
adatok áramlanak. Ez a programozási fajta egyszerűbb mivel nem kell konkrétan a kódokat tudni csak a node-ok funkcióit, így
szélesebb körben is könnyen használható.

A node-RED egyszerűen elérhető egy webböngészőből. Ezen az oldalon megjelenik maga az editor, ahol a bal oldalon megtaláljuk a 
node-okat, itt tudjuk kiválasztani a számunkra szükséges feladatokat amit aztán csak ki kell húznunk a középen lévő üres felületre.
Némelyik node-nak értéket is kell adnunk ezt dupla kattintással megtehetjük amely előhoz egy felületet ahol különböző beállításokat
találunk. Minden programhoz szükségünk van egy bemenetre és egy kimenetre. A programot a jobb felső sarokban lévő Deploy gombbal
tudjuk elindítani ha ez megtörtént akkor a bemenet előtti gomb megnyomásával le is tudjuk futtatni a kódot.

Már előre megírt kódokat is be tudunk tölteni, ha megvan hozzá a JSON fájl, egy egyszerű importálással. Ugyan így menteni is tudjuk
a saját programunkat.

Az első node-RED-es próbálkozás egy egyszerű Hello, World! volt amihez igazából egy input és egy output részre van szükség.
Az input részben megadjuk a stringet ami az üzenetünk és a két részt összekötjük. A Deploy gombra kattintva tudjuk elindítani a
programunkat, ezután pedig az input előtti gombbal be is kapcsoljuk, ekkor megjelenik a Hello, World! felirat.

\begin{figure}[htbp]
	\centering
	\includegraphics[width=0.5\textwidth]{fig/Hello, World.png}
	\caption{Hello, World!}
	\label{fig-Hello, World}
\end{figure}

A második részben több inputot használtam, hogy számokat tudjak bevinni egy outputra. A módszer haszonló mint az előzőnél
csak itt számokat használtam. Működés szempontjából mindig egy számot kaptunk a kimeneten, mindig azt a számot, amely input előtti
gombot megnyomtuk.

\begin{figure}[htbp]
	\centering
	\includegraphics[width=0.5\textwidth]{fig/Numbers.png}
	\caption{Számok bevitele.}
	\label{fig-Numbers}
\end{figure}

A harmadik rész már egy komplexebb program, itt egy UI is van, valamint egy Chart. Az előző programból ismert számos módszer
itt is szerepel viszon ez nem csak kimenetre, hanem egy chartra is rá van kötve. Ha a bemeneten lévő gombok valamelyikét megnyomjuk 
akkor az adott szám megjelenik a grafikonon, ha pedig egyszerre többet megnyomunk akkor egy vonaldiagrammot rajzol nekünk.

\begin{figure}[htbp]
	\centering
	\includegraphics[width=0.5\textwidth]{fig/uichart.png}
	\caption{UI program.}
	\label{fig-uichart}
\end{figure}

\begin{figure}[htbp]
	\centering
	\includegraphics[width=0.5\textwidth]{fig/chart.png}
	\caption{A gombok véletlenszerű megnyomása utáni adatok.}
	\label{fig-chart}
\end{figure}

Az utolsó programhoz már használtam egy fizikai szenzort is. Ez a szenzor képes hőmérsékletet valamint páratartalmat mérni.
Hasonlóan az előzőhöz a számok itt is megvannak azonban itt egy function részen keresztül jutnak el a kimenetre valamint a 
grafikonra. Ez az eszköz a mért adatokból rajzol ki egy diagrammot. A szenzorhoz még csatlakozik egy külön programrész ahol be lehet
állítani, hogy melyik pinen csatlakozik a Raspberry-hez.

\begin{figure}[htbp]
	\centering
	\includegraphics[width=0.5\textwidth]{fig/dht.png}
	\caption{A DHT szenzor programja.}
	\label{fig-dht}
\end{figure}

Sajnos a szenzor több beállítás után sem akart működni.

\chapter{Összefoglalás}

Összességében megismerkedtem a Raspberry PI-al, valamint a node-RED használatával. Ezek segítségével a jövőben szeretném 
továbbfejleszteni a projektmunkám/szakdolgozatom és egy könnyen elérhető kezelőfelületet létrehozni.

\end{document}
