\documentclass[a4paper,12pt,oneside]{report}
\usepackage[magyar]{babel}
\usepackage[utf8]{inputenc}
\usepackage{t1enc}
\def\magyarOptions{defaults=hu-min}
\usepackage{cite}
\usepackage{amsmath,amssymb,amsfonts}
\usepackage{algorithmic}
\usepackage{graphicx}
\usepackage{textcomp}
\usepackage[dvipsnames]{xcolor}
%\usepackage{booktabs}
\usepackage{tabularray}
\usepackage{listings}
\lstset
{
	framexleftmargin=10mm,
	frame=shadowbox,
	rulesepcolor=\color{olive8},
	language=C,
	numbers=left,
	stepnumber=1,
	showstringspaces=false,
	tabsize=4,
	breaklines=true,
	breakatwhitespace=false,
	basicstyle=\footnotesize\ttfamily
}
\frenchspacing
\linespread{1.25}
\UseTblrLibrary{booktabs}
\usepackage[left=2.5cm,top=4cm,right=2.5cm,bottom=4.5cm,headsep=2cm]{geometry}

\usepackage{titlesec, blindtext, color}
\definecolor{gray75}{gray}{0.75}

\usepackage{fancyhdr}

\usepackage{etoolbox}
\patchcmd{\chapter}{\thispagestyle{plain}}{\thispagestyle{fancy}}{}{}

\begin{document}

\pagestyle{fancy}
\fancyhead[]{}
\fancyhead[C]{\thepage}
\fancyfoot[]{}
\fancyfoot[]{}
\renewcommand{\headrulewidth}{0pt}
\renewcommand{\footrulewidth}{0pt}

\titleformat{\chapter}[hang]{\normalsize\bfseries}{\color{Black}\thechapter .\hspace{10pt}}{0pt}{\normalsize\bfseries\color{Black}\MakeUppercase}

\titleformat{\section}[hang]{\normalsize\bfseries}{\hspace{0pt} \thesection .\hspace{10pt}}{0pt}{\normalsize\bfseries\MakeUppercase}

\titleformat{\subsection}[hang]{\normalsize\bfseries}{\hspace{0pt} \thesubsection .\hspace{10pt}}{0pt}{\normalsize\bfseries}

\title{Szakdolgozat}

\author{Példa József}

\date{}

\maketitle

\tableofcontents

\chapter{Bevezetés}

Az ezt megelőző félévekben elkezdett okosotthon projekt folytatása az idei évi is, amely jelenleg egy Raspberry PI-t tartalmaz.
Idén ennek a megismerésével foglalkoztam, valamint a node-REDdel. Ezek bemutatását tartalmazza majd ez a dokumentáció.

%\begin{itemize}

%	\item{Első bullet point}

%	\item{Második bullet point}

%	\item{Végül a harmadik}

%\end{itemize}

%És kódot is. Ilyenkor a LaTeX szépen sorszámozza nekünk, és egy keretet is tesz.
%(A sorszámozás, a keret, a behúzás stb. beállítható.)

%Ez egy kis térköz
\medskip

%Így tudunk kódot beszúrni
\begin{lstlisting}
#include <stdio.h>
int main()
{
	// printf() displays the string inside quotation
	printf("Hello, World!");
	return 0;
}
\end{lstlisting}

\chapter{Telepítés}

A telepítéshez a Raspberry PI Imagert használtam, ami egy eleég egyszerű program.
\Aref{fig-Imager} ábrán egy példa látható.

\begin{figure}[htbp]
	\centering
	\includegraphics[width=0.35\textwidth]{fig/Imager.png}
	\caption{A telepítő program.}
	\label{fig-Imager}
\end{figure}
A programon bellül ki kell választanunk az operációs rendszert amit a Raspberry-n futtatni szeretnénk, ezen kívül a memóriakártyát, 
ami a háttértárunk lesz. Amikor ezzel megvagyunk még a beállításokban meg kell adnunk az eszköznek egy felhasználónevet és egy jelszót
valamint a router hozzáférési adatait.
\Aref{fig-os} ábrán egy példa látható.

\begin{figure}[htbp]
	\centering
	\includegraphics[width=0.35\textwidth]{fig/os.png}
	\caption{A telepítő program.}
	\label{fig-os}
\end{figure}
Ha ezzel megvagyunk akkor elindul egy lassabb telepítési folyamat, amely a memóriakártyára helyezi az operációs rendszert, ezután pedig
egy ellenörzési folyamat, hogy minden rendben van-e.
Miután ezzel megvagyunk a memóriakártyát behelyezzük a Raspberry-be, ahol egy lassabb boot után elindul az eszköz (amikor ez megtörtént
akkor a led egy biztos ritmusra villog), ezután megkezdhetjük a hozzáférést az eszközhöz. 
A hozzáféréshez szükségünk van a router kezelőfelületére, ahol meg kell keresnünk az eszközünket, a legjobb ha adunk neki
egy saját nevet. Miután ez megvan a PuTTY segítségével hozzáférünk a Raspberry operációs rendszeréhez.
\Aref{fig-putty} ábrán egy példa látható.

\begin{figure}[htbp]
	\centering
	\includegraphics[width=0.35\textwidth]{fig/putty.png}
	\caption{A telepítő program.}
	\label{fig-putty}
\end{figure}

\chapter{Rendszerterv}

%\section{Azon belül egy alfejezet}

%\subsection{Még egy szinttel beljebb}

%\section{Újabb alfejezet}

\chapter{Megvalósítás}


\chapter{Összefoglalás}


\end{document}
